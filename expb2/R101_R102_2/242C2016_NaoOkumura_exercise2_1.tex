\documentclass[10pt,a4paper,titlepage]{jreport} % ドキュメントクラスを1つに統一


\usepackage[top=25truemm,bottom=25truemm,left=20truemm,right=20truemm]{geometry} % 余白設定
\usepackage[dvipdfmx]{graphicx} % 画像の挿入
\usepackage{graphicx}
\usepackage{listings} % コードリスト用
\usepackage{jlisting} % 日本語対応のリスト用
\usepackage{jverb} % 日本語のベタ打ち
\usepackage{booktabs}
\usepackage{pgfplots} % グラフ描画用パッケージ
\pgfplotsset{compat=1.18} % バージョン指定

\parindent = 0pt % 段落の字下げを無効化

\usepackage{titlesec}
\usepackage{etoolbox}
\makeatletter
\patchcmd{\chapter}{\if@openleft\cleardoublepage\else\if@openright\cleardoublepage\else\clearpage\fi\fi}{}{}{}
\makeatother
\lstset{breaklines=true, postbreak=\mbox{$\hookrightarrow$}\space, keepspaces=true, escapeinside={\%*}{*)}}

\titleformat{\chapter}[hang] % 章のフォーマットを変更
  {\normalfont\huge\bfseries} % フォント設定
  {\thechapter} % 番号部分の表示形式
  {1em} % 番号とタイトルの間のスペース
  {} % タイトルの前に入れる内容(ここでは空)
\usepackage{amsmath}
\usepackage{xcolor}

\lstset{
  language=Matlab,              % 言語はMATLAB
  basicstyle=\ttfamily\small,   % フォントは等幅、サイズ小さめ
  keywordstyle=\color{blue},    % キーワード(function, endとか)を青
  commentstyle=\color{green!50!black}, % コメントを深緑
  stringstyle=\color{red!70!black},    % 文字列('...')を赤系
  numbers=left,                 % 行番号を左に表示
  numberstyle=\tiny\color{gray},% 行番号をグレーで小さく
  stepnumber=1,                 % 毎行行番号つける
  numbersep=10pt,               % コードとの間隔
  backgroundcolor=\color{gray!10}, % 背景うっすらグレー
  frame=single,                 % 枠をつける
  rulecolor=\color{black},      % 枠線は黒
  breaklines=true,              % 長い行を折り返す
  breakatwhitespace=false,      % スペースでだけ折り返すか(falseにして自然な折り返し)
  showspaces=false,             % スペースは特別に表示しない
  showstringspaces=false,       % 文字列中のスペースも普通に表示
  tabsize=2                     % タブ幅2
}

\title{R101/R102 演習2-1} % タイトル
\author{
  学生番号:242C2016  氏名:奥村直 \\
  \\
  知的システム工学科システム制御コース
  } % 著者
\date{\today} % 日付

\begin{document}

\maketitle

\chapter{ソースコード}
\begin{lstlisting}[caption= digital_pi.m]
function [vi, desired_rev] = digital_pi(t, omega, get_reference_fn, Kp, Ki)
% Customizable digital PI controller
%
% Inputs:
%   t - Current time [s]
%   omega - Current angular velocity [rad/s]
%   get_reference_fn - Function handle that returns desired revolution [rps]
%   Kp - Proportional gain
%   Ki - Integral gain
%
% Outputs:
%   vi - Calculated input voltage [V]
%   desired_rev - Desired revolution [rps]

    % Get the desired revolution at the current time point
    desired_rev = get_reference_fn(t);

    % Convert desired revolution (rps) to angular velocity (rad/s)
    desired_omega = desired_rev * 2 * pi;

    % Compute control error (rad/s)
    err = desired_omega - omega;

    % Persistent storage for integral computation
    persistent acc_err_storage last_time_storage;
    if isempty(acc_err_storage)
        acc_err_storage = containers.Map('KeyType', 'char', 'ValueType', 'any');
        last_time_storage = containers.Map('KeyType', 'char', 'ValueType', 'any');
    end

    % Create a unique key for each Kp/Ki pair
    key = sprintf('%.5f_%.5f', Kp, Ki);

    % Initialize accumulated error and last time if first call
    if t == 0 || ~isKey(acc_err_storage, key)
        acc_err_storage(key) = 0;
        last_time_storage(key) = t;
    end

    % Time step since last call
    dt = t - last_time_storage(key);
    last_time_storage(key) = t;

    % Update integral term (accumulated error)
    acc_err = acc_err_storage(key) + err * dt;
    acc_err_storage(key) = acc_err;

    % PI control law
    vi = Kp * err + Ki * acc_err;
end

\end{lstlisting}

\chapter{参考文献・生成AI}
- テキスト(第1章まで)\\
- ChatGPT 4o
\end{document}