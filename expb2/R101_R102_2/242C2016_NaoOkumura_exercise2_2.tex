\documentclass[10pt,a4paper,titlepage]{jreport} % ドキュメントクラスを1つに統一


\usepackage[top=25truemm,bottom=25truemm,left=20truemm,right=20truemm]{geometry} % 余白設定
\usepackage[dvipdfmx]{graphicx} % 画像の挿入
\usepackage{graphicx}
\usepackage{listings} % コードリスト用
\usepackage{jlisting} % 日本語対応のリスト用
\usepackage{jverb} % 日本語のベタ打ち
\usepackage{booktabs}
\usepackage{pgfplots} % グラフ描画用パッケージ
\pgfplotsset{compat=1.18} % バージョン指定

\parindent = 0pt % 段落の字下げを無効化

\usepackage{titlesec}
\usepackage{etoolbox}
\makeatletter
\patchcmd{\chapter}{\if@openleft\cleardoublepage\else\if@openright\cleardoublepage\else\clearpage\fi\fi}{}{}{}
\makeatother
\lstset{breaklines=true, postbreak=\mbox{$\hookrightarrow$}\space, keepspaces=true, escapeinside={\%*}{*)}}

\titleformat{\chapter}[hang] % 章のフォーマットを変更
  {\normalfont\huge\bfseries} % フォント設定
  {\thechapter} % 番号部分の表示形式
  {1em} % 番号とタイトルの間のスペース
  {} % タイトルの前に入れる内容(ここでは空)
\usepackage{amsmath}
\usepackage{xcolor}

\lstset{
  language=Matlab,              % 言語はMATLAB
  basicstyle=\ttfamily\small,   % フォントは等幅、サイズ小さめ
  keywordstyle=\color{blue},    % キーワード(function, endとか)を青
  commentstyle=\color{green!50!black}, % コメントを深緑
  stringstyle=\color{red!70!black},    % 文字列('...')を赤系
  numbers=left,                 % 行番号を左に表示
  numberstyle=\tiny\color{gray},% 行番号をグレーで小さく
  stepnumber=1,                 % 毎行行番号つける
  numbersep=10pt,               % コードとの間隔
  backgroundcolor=\color{gray!10}, % 背景うっすらグレー
  frame=single,                 % 枠をつける
  rulecolor=\color{black},      % 枠線は黒
  breaklines=true,              % 長い行を折り返す
  breakatwhitespace=false,      % スペースでだけ折り返すか(falseにして自然な折り返し)
  showspaces=false,             % スペースは特別に表示しない
  showstringspaces=false,       % 文字列中のスペースも普通に表示
  tabsize=2                     % タブ幅2
}

\title{R101/R102 演習2-2} % タイトル
\author{
  学生番号:242C2016  氏名:奥村直 \\
  \\
  知的システム工学科システム制御コース
  } % 著者
\date{\today} % 日付


\begin{document}
\maketitle
\chapter{ソースコード}
\begin{lstlisting}[caption=exercise2-1]
  %% PI Parameter Comparison Simulation
clear variables;

% Simulation time settings
tspan = 0:1e-4:5e-2; % 0~0.05 seconds with 0.0001-second intervals

% Initial state [current A; angular velocity rad/s]
x_init = [0.0; 0.0];

% Generate reference trajectory (constant, sine, step, or ramp)
reference_pattern = 'constant'; % Choose from: 'constant', 'sine', 'step', 'ramp'
ref_trajectory = generate_reference(tspan, reference_pattern);

% PI parameter sets to compare
pi_params = [
    1.0, 0.02;    % Kp = 0.1, Ki = 0.0 (P control only)
    1.0, 0.015;    % Kp = 0.5, Ki = 0.0 (P control only)
];

% Cell array to store results
collectors = cell(size(pi_params, 1), 1);

% ODE45 options
options = odeset('AbsTol', 1e-6, 'RelTol', 1e-3);

% Run simulation for each PI parameter set
  for i = 1:size(pi_params, 1)
    % Get current PI parameters
    Kp = pi_params(i, 1);
    Ki = pi_params(i, 2);
    
    % Create reference function for interpolation
    reference_fn = @(t) interp1(tspan, ref_trajectory, t, 'linear', 'extrap');
    
    % Create input voltage function (using custom PI controller)
    input_voltage_fn = @(t, omega) digital_pi(t, omega, reference_fn, Kp, Ki);
    
    % Simulate DC motor dynamics
    [t, x] = ode45(@(t, x) dynamics_wrapper(t, x, input_voltage_fn, @load_torque),...
        tspan, x_init, options);
    
    % Create data collector and process results
    collector = DataCollector();
    collector.process(t, x, input_voltage_fn, @load_torque);
    collector.setParameters(Kp, Ki);
    
    % Store results
    collectors{i} = collector;
    
    fprintf('Simulation %d/%d completed: %s\n', i, size(pi_params, 1), collector.description);
  end

    % Display comparison results - using DataCollector's static method
    DataCollector.plotPIComparisonResults(collectors, reference_pattern);
\end{lstlisting}

$K\textsubscript{p}$および$K\textsubscript{i}$はソースコードの$pi_params$に示した通りである。

\chapter{結果のグラフと考察}

\section{グラフ}
\begin{figure}[htbp]
  \centering
  \includegraphics[width=0.5\textwidth]{exercise2_2.eps}
  \caption{目標回転数に収束するようにゲインを調整したグラフ}
  \label{fig:epsimage}
\end{figure}

\section{考察}
指定された$K\textsubscript{p}$および$K\textsubscript{i}$の制約条件の中では、目標回転数の$1rps$に到達すること
ができなかった。回転数が最も速く滑らかに収束する$K\textsubscript{p}$と$K\textsubscript{i}$の値を用いてグラフ
を描画した。何回も数値を変更してはグラフを出力して、試したがうまく行かなかった。$K\textsubscript{p}$の値を大きくすると
収束する回転数も大きくなった。また、$K\textsubscript{i}$をいくら調整したとしても制御誤差、偏差を排除することはできなかった。

\chapter{参考文献・生成AI}
- テキスト(第2章まで)\\
- ChatGPT 4o
\end{document}